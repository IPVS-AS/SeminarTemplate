% This template has been tested with LLNCS DOCUMENT CLASS -- version 2.21 (12-Jan-2022)

% !TeX spellcheck = en-US
% !TeX encoding = utf8
% !TeX program = pdflatex
% !BIB program = bibtex
% -*- coding:utf-8 mod:LaTeX -*-

% "a4paper" enables:
%
%  - easy print out on DIN A4 paper size
%
% One can configure a4 vs. letter in the LaTeX installation. So it is configuration dependend, what the paper size will be.
% This option  present, because the current word template offered by Springer is DIN A4.
% We accept that DIN A4 cause WTFs at persons not used to A4 in USA.
%
% "runningheads" enables:
%
%  - page number on page 2 onwards
%  - title/authors on even/odd pages
%
% This is good for other readers to enable proper archiving among other papers and pointing to
% content. Even if the title page states the title, when printed and stored in a folder, when
% blindly opening the folder, one could hit not the title page, but an arbitrary page. Therefore,
% it is good to have title printed on the pages, too.
%
% To disable outputting page headers and footers, remove "runningheads"
\documentclass[runningheads,a4paper]{llncs}[2022/01/12]


%%%%%%%%%%%%%%%%%%%%%%%%%%%%%%%%%%%%%%%%%%%%%%%%%%%%%%%%%%%%%%%%%%%%%%%%%%%%%%%
% Language configuration                                                      %
%%%%%%%%%%%%%%%%%%%%%%%%%%%%%%%%%%%%%%%%%%%%%%%%%%%%%%%%%%%%%%%%%%%%%%%%%%%%%%%
% Language that is supposed to be used
\newif\ifgerman
\germantrue % Use for German
%\germanfalse % Use for English

% Correct bad hyphenation
\hyphenation{op-tical net-works semi-conduc-tor}


%%%%%%%%%%%%%%%%%%%%%%%%%%%%%%%%%%%%%%%%%%%%%%%%%%%%%%%%%%%%%%%%%%%%%%%%%%%%%%%
% General configuration                                                       %
%%%%%%%%%%%%%%%%%%%%%%%%%%%%%%%%%%%%%%%%%%%%%%%%%%%%%%%%%%%%%%%%%%%%%%%%%%%%%%%
% Load required packages and configure their usage
\input{config}

% Set graphics path
\graphicspath{{graphics/}}


%%%%%%%%%%%%%%%%%%%%%%%%%%%%%%%%%%%%%%%%%%%%%%%%%%%%%%%%%%%%%%%%%%%%%%%%%%%%%%%
% Seminar paper configuration                                                 %
%%%%%%%%%%%%%%%%%%%%%%%%%%%%%%%%%%%%%%%%%%%%%%%%%%%%%%%%%%%%%%%%%%%%%%%%%%%%%%%
% Cover page and title
\usepackage[
	title={My seminar paper}, % Title of the seminar paper
	%shortTitle={Short title}, % Short title, only needed if title exceeds one line in the running headlines
	author={Forename Surname}, % Author of the seminar paper
	email={st123456@stud.uni-stuttgart.de}, % Email address of the author
	seminarName={Seminar name}, % Name or topic of the seminar
	semester={WS 2022/2023}, % Semester of the seminar course
	deadline={Stuttgart, 17.05.2022}, % Deadline for the seminar paper
	supervisor={Forename Surname}, % Name of the supervisor
	linkDate={17.05.2022}, % Date when the links were checked for the last time
	isGerman={\ifgerman true \else false \fi} % Whether German is used instead of English
]{lncs_as}


%%%%%%%%%%%%%%%%%%%%%%%%%%%%%%%%%%%%%%%%%%%%%%%%%%%%%%%%%%%%%%%%%%%%%%%%%%%%%%%
% Document start                                                              %
%%%%%%%%%%%%%%%%%%%%%%%%%%%%%%%%%%%%%%%%%%%%%%%%%%%%%%%%%%%%%%%%%%%%%%%%%%%%%%%
\begin{document}

% Create cover page and title
\CoverPage
\Title


%%%%%%%%%%%%%%%%%%%%%%%%%%%%%%%%%%%%%%%%%%%%%%%%%%%%%%%%%%%%%%%%%%%%%%%%%%%%%%%
% Content start                                                               %
%%%%%%%%%%%%%%%%%%%%%%%%%%%%%%%%%%%%%%%%%%%%%%%%%%%%%%%%%%%%%%%%%%%%%%%%%%%%%%%

\begin{abstract}
Die Kurzfassung sollte den Inhalt der Arbeit kurz wiedergeben und aus mindestens 70 und maximal 150 Wörtern bestehen.
Sie wird in einer 9-Punkt Schrift gesetzt und 1 cm eingerückt von rechts und links.
Vor und nach der Kurzfassung sind zwei leere Zeilen.
\end{abstract}

\begin{keywords}
Keyword1, Keyword2
\end{keywords}


\section{Einleitung}
Ab diesem Abschnitt fängt die eigentliche Arbeit an und man beginnt mit einer Einführung und der Einordnung des Themas.

Absätze werden durch Leerzeilen erreicht.

Als \LaTeX-Editor sind TeXstudio\footnote{\url{http://texstudio.sourceforge.net/}} oder TeXlipse\footnote{\url{http://texlipse.sourceforge.net/}} empfehlenswert.

Für die Versionskontrolle bietet es sich an, jeden Satz in einer neuen Zeile zu beginnen.

Als Nachschlagewerk und gute Einführung in das Thema \LaTeX\ bietet sich \emph{The Not So Short Introduction to \LaTeXe}\ von Oetiker et al an.
Auf der Seite \url{http://tobi.oetiker.ch/lshort/} ist dieses Dokument zum Herunterladen bereitgestellt.

Ein weiteres, empfehlenswertes Dokument ist das \emph{Kochbuch für \LaTeX{}}, welches auf der Seite \url{http://www.uni-giessen.de/hrz/tex/cookbook/cookbook.html} zu finden ist.

Eine Vorlage für Bachelor- und Masterarbeiten ist unter \url{https://github.com/latextemplates/uni-stuttgart-computer-science-template} zu finden.

Für weitere Anleitungen und Informationen empfiehlt sich die Homepage \url{http://www.dante.de/}.
Des Weiteren gibt es eine Reihe von Literatur zum Thema \LaTeX\ von Bürger~\cite{buerger}, Kopka~\cite{kopka}, Günther~\cite{guenther} und vielen anderen.
Weitere Hinweise zu \LaTeX{} unter \url{http://wiki.flupp.de/latex}.
Zu Ausarbeitungen im Allgemeinen unter \url{http://wiki.flupp.de/studium/ausarbeitungen}.

Die Arbeit gleidert sich wie folgt:
...
Schließlich wird in \cref{sec:zusfas} eine Zusammenfassung der Arbeit und ein Ausblick auf weitere Forschungsthemen gegeben.

\subsection{Beispiel für Überschrift 2.~Ebene}

\subsubsection{Beispiel für Überschrift 3.~Ebene}

\paragraph{Beispiel für Überschrift 4.~Ebene}
Dies ist Beispielabsatz zu einem Beispiel einer Überschrift 4.~Ebene.

\section{Formatierungsrichtlinien}

\subsection{Tabellen}
In diesem Abschnitt wird das Einfügen einer Tabelle und das Füllen derselben mit Werten gezeigt.
Im Gegensatz zu Abbildungen erfolgt die Beschriftung üblicherweise oberhalb und nicht unterhalb der Tabelle.
Der letzte Satz der Beschriftung endet ohne Punkt.
Die \Cref{tab:bsp} zeigt die Größenentwicklung der Weltbevölkerung von 8000 v.\,Chr. bis 1980 n.\,Chr.

\begin{table}
\caption{Dieses Beispiel entstammt dem {\it\TeX{}Buch,} S.\,246}
\label{tab:bsp}
\begin{center}
\begin{tabular}{r@{\quad}rl}
\hline
\multicolumn{1}{l}{\rule{0pt}{12pt}
                   Jahr}&\multicolumn{2}{l}{Weltbevölkerung}\\[2pt]
\hline\rule{0pt}{12pt}
8000 v.\,Chr. &     5,000,000& \\
  50 n\,Chr. &   200,000,000& \\
1650 n\,Chr. &   500,000,000& \\
1945 n\,Chr. & 2,300,000,000& \\
1980 n\,Chr. & 4,400,000,000& \\[2pt]
\hline
\end{tabular}
\end{center}
\end{table}

\subsection{Code-Beispiele}

\Cref{lst:example} zeigt ein Code-Beispiel.

\begin{lstlisting}[float,caption=A floating example,label=lst:example]
public static void main(String[] args) {
}
\end{lstlisting}

\subsection{Formeln}
Formeln sind in aufsteigender Reihenfolge, beginnend mit der Nummer~1, durchzunummerieren.
Dies erleichtert den Bezug vom Text auf die zugehörige Formel.
Die Formel
%
\begin{equation}
  \psi (u) = \int_{o}^{T} \left[\frac{1}{2}
  \left(\Lambda_{o}^{-1} u,u\right) + N^{\ast} (-u)\right] dt
\end{equation}
%
berechnet etwas.
Formeln sind bezüglich der Zeichensetzung wie normaler Text zu behandeln, d.\,h.\ falls ein Satz mit einer Formel endet, folgt dieser ein Punkt, der üblicherweise durch ein schmales Leerzeichen abgesetzt wird.

\subsection{Abbildungen}
Im nachfolgenden Beispiel wird dargestellt, wie Grafiken eingefügt werden und wie die Beschriftung bzw.\ die Formatierung derselben auszusehen hat.

Bei Abbildungen erfolgt die Beschriftung unterhalb der Abbildung (siehe \cref{fig:logo}).
Der letzte Satz der Beschreibung endet wie bei Tabellenbeschriftungen ohne Punkt.
Um eine gute Qualität zu erzielen, muss die Auflösung der Abbildungen mindestens 300~DPI betragen.
Bei Strichzeichnungen empfiehlt sich sogar eine weit höhere Auflösung, zwischen 800 und
1200~DPI.

Am besten Vektorgraphiken nehmen und als \texttt{.pdf} exportieren.

\begin{figure}
  \begin{center}
    \includegraphics[width=.5\textwidth]{ipvs_logo.png}
    \caption{Das Logo des Instituts}
    \label{fig:logo}
   \end{center}
\end{figure}

\subsection{Literaturverweise}
Ein Verweis auf eine im Literaturverzeichnis aufgeführte
Literaturquelle wird durch die in eckige Klammern eingeschlossene
Nummer der entsprechenden Literaturquelle angegeben. Als Beispiel ist
hier nochmals auf Buerger~\cite{buerger} verwiesen.

\subsection{Anführungszeichen}
Entweder "`so"' tippen oder \enquote{so}.
``So'' bitte nicht, das führt zu englischen Anführungszeichen.

\subsection{Sonstiges}
Brackets work as designed:
<test>

The symbol for powerset is now correct: $\powerset$ and not a Weierstrass p ($\wp$).

\begin{inparaenum}
\item All these items...
\item ...appear in one line
\item This is enabled by the paralist package.
\end{inparaenum}

``something in quotes'' using plain tex or use \enquote{the enquote command}.

You can now write words containing hyphens which are hyphenated (application"=specific) at other places.
This is enabled by an additional configuration of the babel package.
In case you write \enquote{application-specific}, then the word will only be hyphenated at the dash.
You can also write applica\allowbreak{}tion-specific, but this is much more effort.

\section{Zusammenfassung}
\label{sec:zusfas}
Dieser Absatz schließt die Arbeit ab.

\subsubsection*{Acknowledgments}
...

In the bibliography, use \texttt{\textbackslash textsuperscript} for ``st'', ``nd'', ...:
E.g., \enquote{The 2\textsuperscript{nd} conference on examples}.
When you use \href{https://www.jabref.org}{JabRef}, you can use the clean up command to achieve that.
See \url{https://help.jabref.org/en/CleanupEntries} for an overview of the cleanup functionality.

%%%%%%%%%%%%%%%%%%%%%%%%%%%%%%%%%%%%%%%%%%%%%%%%%%%%%%%%%%%%%%%%%%%%%%%%%%%%%%%
% Content end                                                                 %
%%%%%%%%%%%%%%%%%%%%%%%%%%%%%%%%%%%%%%%%%%%%%%%%%%%%%%%%%%%%%%%%%%%%%%%%%%%%%%%

\BibliographySection
\end{document}

%%%%%%%%%%%%%%%%%%%%%%%%%%%%%%%%%%%%%%%%%%%%%%%%%%%%%%%%%%%%%%%%%%%%%%%%%%%%%%%
% Document end                                                                %
%%%%%%%%%%%%%%%%%%%%%%%%%%%%%%%%%%%%%%%%%%%%%%%%%%%%%%%%%%%%%%%%%%%%%%%%%%%%%%%
